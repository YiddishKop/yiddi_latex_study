\documentclass[12pt, a4paper, addpoints]{exam} % addpoints: 允许在每题后指定分数

% 1. 中文字体和页面设置
\usepackage{xeCJK}
\usepackage{exam}
\setCJKmainfont{Source Han Serif SC} % 使用思源宋体,请确保系统已安装
\usepackage[inner=2cm, outer=2cm, top=2cm, bottom=2cm]{geometry}
\usepackage{amsmath, amssymb} % 数学公式和符号

% --- 关键修改:将分数单位从 "points" 改为 "分" ---
\renewcommand{\pointsofdigit}[1]{#1分}
\renewcommand{\pointsofchar}[1]{#1分}
% ----------------------------------------------------

% 2. 自定义作业信息
\def\TitleName{高一数学课后作业 (函数与方程)}
\def\TeacherName{李老师}
\def\chinesedate{\the\year 年\the\month 月\the\day 日}
\def\ClassName{高一(1)班}

% 3. 页眉和页脚设置
\pagestyle{headandfoot} % 启用 exam 宏包的页眉页脚

% 页眉内容(\small 是为了让字体小一点)
\lhead{\small \TitleName}               % 左页眉:作业名称
\chead{\small 授课教师:\TeacherName}    % 中页眉:教师
\rhead{\small 班级:\ClassName}         % 右页眉:班级
\footer{}{\thepage}{}                   % 页脚:只在中间显示页码

% 页眉下方横线和间距
\renewcommand{\headrule}{%
  \hrule height 0.4pt
  \vspace{6pt} % 页眉和正文间距
}


\begin{document}

% 顶部信息和总分
\noindent\textbf{姓名:\qquad\qquad\qquad\qquad 成绩:\rule{3cm}{0.4pt}}
\par\vspace{0.3cm}
\par\vspace{0.5cm}

\begin{questions}

% --- 题目开始 ---

% 显示为 (10分)
\question[10] 已知集合 $A=\{x \mid x^2-3x+2<0\}$ 和 $B=\{x \mid x>1\}$。
求 $A \cap B$。
\vspace{4cm}

% 总分显示为 (15分)
\question[15] 求函数 $f(x) = \sqrt{4-x^2}$ 的定义域和值域。
\begin{parts}
    % 小题显示为 (8分) 和 (7分)
    \part[8] 求函数的定义域。
    \part[7] 求函数的值域。
\end{parts}
\vspace{6cm}

% 显示为 (10分)
\question[10] (单选题) 下列函数中,在 $(0, +\infty)$ 上单调递增的是:
\begin{choices}
    \choice $y = x^2$
    \choice $y = -x+1$
    \choice $y = \frac{1}{x}$
    \choice $y = \log_{0.5} x$
\end{choices}
\vspace{2cm}

% 显示为 (20分)
\question[20] (解答题) 某工厂生产一种产品,成本 $C(x)=40x+100$,售价 $P(x)=100x-x^2$,其中 $x$ 为产量(百件)。求当产量为多少时,工厂的利润最大?
\vspace{8cm}

\end{questions}

\end{document}