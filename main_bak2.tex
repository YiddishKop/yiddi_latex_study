\documentclass[a4paper,fontsize=8pt]{kaobook}

%------------------------------------------
%       Packages
%------------------------------------------
\usepackage{amsmath,amssymb} % Math packages
\usepackage{physics} % Physics package
\usepackage{tcolorbox} % For colored boxes
\usepackage{xcolor}
\usepackage[UTF8]{ctex}
\usepackage[framed, theorembackground=blue!10]{kaotheorems}
%\usepackage{kaotheorems} % Theorem environments
\setCJKmainfont{Source Han Serif SC}

%------------------------------------------
%==Colour Palette==
%------------------------------------------
\definecolor{merah}{HTML}{F4564E}
\definecolor{merahtua}{HTML}{89313E}
\definecolor{biru}{HTML}{60BBE5}
\definecolor{birutua}{HTML}{412F66}
\definecolor{hijau}{HTML}{59CC78}
\definecolor{hijautua}{HTML}{366D5B}
\definecolor{kuning}{HTML}{FFD56B}
\definecolor{jingga}{HTML}{FBA15F}
\definecolor{ungu}{HTML}{8C5FBF}
\definecolor{lavender}{HTML}{CBA5E8}
\definecolor{merjamb}{HTML}{FFB6E0}


\definecolor{1}{RGB}{163,187,219}
\definecolor{2}{RGB}{239,239,239}

\definecolor{3}{RGB}{65,222,183}
\definecolor{4}{RGB}{237,241,187}

\definecolor{5}{RGB}{129,227,247}
\definecolor{6}{RGB}{236,239,244}

\definecolor{7}{RGB}{195,217,78}
\definecolor{8}{RGB}{245,243,242}

\definecolor{9}{RGB}{240,145,161}
\definecolor{10}{RGB}{245,242,233}


\definecolor{mycolor1}{RGB}{185,227,251}
\definecolor{mycolor2}{RGB}{252,204,203}
\definecolor{mycolor3}{RGB}{250,230,233}
\definecolor{mycolor4}{RGB}{220,238,248}
\definecolor{mycolor5}{RGB}{131,203,172}
\definecolor{mycolor6}{RGB}{185,222,201}
\definecolor{mycolor7}{RGB}{233,215,223}
\definecolor{mycolor8}{RGB}{200,173,196}
\definecolor{mycolor9}{RGB}{92,197,204}
\definecolor{mycolor10}{RGB}{236,138,164}




%------------------------------------------
%      Document Information 
%------------------------------------------ 
\title{yiddishkop latex 学习一 --- 数学表达式}
\author{John Doe}
\date{\today}


%------------------------------------------
%      Document
%------------------------------------------

\begin{document}

\maketitle
\chapter{数学公式}

\section{三种基本公式}
这是一个行内公式: $E=mc^2$

这是一个居中公式:
\[
    E=mc^2
\]

这是一个带编号的居中公式:
\begin{equation}
    E=mc^2
\end{equation}

\section{多行公式对齐}

解方程过程:
\begin{align}
2x + 3y &= 6 \\
2x &=4\\
x &=2
\end{align}

二项式公式:
\begin{align}
    (a+b)^2 &= a^2 + 2ab + b^2 \\
    (a-b)^2 &= a^2 - 2ab + b^2 \\
    (a+b)(a-b) &= a^2 - b^2
\end{align}



\section{居中对齐的多行公式}

\begin{gather}
    E=mc^2 \\
    F=ma \\
    a^2 + b^2 = c^2
\end{gather}


\section{多列对齐的公式}

\begin{alignat}{2}
    x &= y & \quad & \text{(由条件1)} \\
    a &= b &       & \text{(由条件2)} \\
    m &= n &       & \text{(由条件3)}
\end{alignat}



\section{矩阵表示}

无括号矩阵:
\[
\begin{matrix}
    a & b \\
    c & d
\end{matrix}
\]

圆括号矩阵:
\[
\begin{pmatrix}
    a & b \\
    c & d
\end{pmatrix}
\]

方括号矩阵:
\[
\begin{bmatrix}
    a & b \\
    c & d  
\end{bmatrix}
\]


行列式:
\[
\begin{vmatrix}
    a & b \\
    c & d
\end{vmatrix}
= ad - bc
\]

大括号矩阵:
\[
\begin{Bmatrix}
    a & b \\
    c & d
\end{Bmatrix}
\]


\section{分块矩阵}
\[
\left[
\begin{array}{cc|c}
    1 & 2 & 3 \\
    4 & 5 & 6 \\
    \hline
    7 & 8 & 9
\end{array}
\right]
\]


或者是用 pmatrix 环境:
\[
\begin{pmatrix}
    A & B \\
    C & D  
\end{pmatrix}
\]



\section{分段函数}

绝对值函数:
\[
|x| = 
\begin{cases}
    x & \text{如果 } x \geq 0 \\
    -x & \text{如果 } x < 0
\end{cases}
\]

符号函数:
\[
\operatorname{sgn}(x) =
\begin{cases}
    1 & \text{当 } x > 0 \\
    0 & \text{当 } x = 0 \\
    -1 & \text{当 } x < 0
\end{cases}
\]

阶梯函数:
\[
f(x) =
\begin{cases}
    x^2 & \text{对于} x \leq 0 \\
    \sqrt{x} & \text{对于 } x > 0
\end{cases}
\]



\section{定理环境使用}

这是一个定义:
\begin{definition}[平行四边形]
    如果一个四边形的两组对边分别平行,那么这个四边形就是一个平行四边形。
\end{definition}


这是一个定理:
\begin{theorem}[平行四边形性质]
    在平行中:
    \begin{enumerate}
        \item 对边相等
        \item 对角相等
        \item 对角线互相平分
    \end{enumerate}
\end{theorem}


这是一个证明:
\begin{proof}
    设平行四边形 $ABCD$, 连接对角线 $AC$.
    在 $\triangle ABC$ 和 $\triangle CDA$ 中:
    \begin{align*}
        \angle BAC &= \angle DCA & \text{(对顶角)} \\
        \angle BCA &= \angle CDA & \text{(对顶角)} \\
        AB &= CD & \text{(对边相等)}
    \end{align*}

    因此 $\triangle ABC \cong \triangle CDA$ (ASA), 所以 $AC$ 平分 $\angle BAD$ 和 $\angle BCD$.
\end{proof}

这是一个证明(使用 alignat)
\begin{proof}
    设平行四边形 $ABCD$, 连接对角线 $AC$.
    在 $\triangle ABC$ 和 $\triangle CDA$ 中:
    \begin{alignat}{2}
        \angle BAC &= \angle DCA & \quad &\text{(对顶角)} \\
        \angle BCA &= \angle CDA &       &\text{(对顶角)} \\
        AB &= CD &                       &\text{(对边相等)}
    \end{alignat}

    因此 $\triangle ABC \cong \triangle CDA$ (ASA), 所以 $AC$ 平分 $\angle BAD$ 和 $\angle BCD$.
\end{proof}



这是一个例子:
\begin{example}
    设 $ABCD$ 是一个平行四边形,证明 $AB \parallel CD$.
\end{example}


这是一个答案:
\begin{theorem}
    由于 $ABCD$ 是平行四边形,定义中已经说明 $AB \parallel CD$.
    \begin{align*}
        \angle C &= \angle A & \text{(对顶角)} \\
        \angle B &= \angle D=180^\circ -65^\circ=115^\circ & \text{(对顶角)}
    \end{align*}
\end{theorem}


\chapter{数学符号和特殊字体}

\section{数学字体}
普通字体: $a, b, c, x, y, z$

黑板粗体: $\mathbb{ABC}$ 用于数集: $\mathbb{R}, \mathbb{X}, \mathbb{Z}, \mathbb{N}, \mathbb{Q}, \mathbb{C}$

手写体:$\mathcal{ABC}$, $\mathscr{ABC}$

无衬线体:$\mathsf{ABCabcxyz}$

打字机体:$\mathtt{ABCabcxyz}$


\section{常用数学符号}
数集:$\mathbb{N} \subset \mathbb{Z} \subset \mathbb{Q} \subset \mathbb{R} \subset \mathbb{C}$

运算符:$\pm, \mp, \times, \div, \cdot, \ast, \star$

关系符:$\sim, \approx, \simeq, \cong, \equiv, \propto$

箭头:$\to, \rightarrow, \leftarrow, \Rightarrow, \Leftrightarrow, \mapsto$

几何符号:$\angle, \triangle, \square, \parallel, \perp, \cong$


\begin{tcolorbox}[enhanced,colback=gray!5,colframe=gray!75,breakable,coltitle=green!25!black,title=2024]
 设 $ \boldsymbol{x}^{(k)} \in \mathbf{R}^{n}, k=0,1,2, \cdots, \boldsymbol{x}^{*} \in \mathbf{R}^{n}, \boldsymbol{B} \in \mathbf{R}^{n \times n} $.

(1) 给出向量序列 $ \boldsymbol{x}^{(k)}(k=0,1,2, \cdots) $ 收敛于向量 $ \boldsymbol{x}^{*} $ 的定义;

(2) 设 $ \lim\limits _{k \rightarrow \infty} \boldsymbol{x}^{(k)}=\boldsymbol{x}^{*} $, 证明: $ \lim\limits _{k \rightarrow \infty} B \boldsymbol{x}^{(k)}=\boldsymbol{B} \boldsymbol{x}^{*} $.

 \tcblower

设 $ \|\cdot\| $ 为 $ \mathbf{R}^{n} $ 中的一种范数.

(1) 如果
$$
\lim _{k \rightarrow \infty}\left\|\boldsymbol{x}^{(k)}-x^{*}\right\|=0,
$$
则称向量序列 $ \left\{\boldsymbol{x}^{(k)}\right\}_{k=0}^{\infty} $ 收敛于向量 $ \boldsymbol{x}^{*} $.

(2) 因 $ \lim\limits _{k \rightarrow \infty} x^{(k)}=x^{*} $, 则 $ \lim\limits _{k \rightarrow \infty}\left\|x^{(k)}-x^{*}\right\|=0 $. 又
$$
\left\|\boldsymbol{B} \boldsymbol{x}^{(k)}-\boldsymbol{B} \boldsymbol{x}^{*}\right\|=\left\|\boldsymbol{B}\left(\boldsymbol{x}^{(k)}-\boldsymbol{x}^{*}\right)\right\| \leqslant\|\boldsymbol{B}\| \cdot\left\|\boldsymbol{x}^{(k)}-\boldsymbol{x}^{*}\right\|,
$$
所以
$$
\begin{aligned}
\lim _{k \rightarrow \infty}\left\|\boldsymbol{B} \boldsymbol{x}^{(k)}-\boldsymbol{B} \boldsymbol{x}^{*}\right\| & \leqslant \lim _{k \rightarrow \infty}\|\boldsymbol{B}\| \cdot\left\|\boldsymbol{x}^{(k)}-\boldsymbol{x}^{*}\right\| \\
& =\|\boldsymbol{B}\| \lim _{k \rightarrow \infty}\left\|\boldsymbol{x}^{(k)}-\boldsymbol{x}^{*}\right\|=0,
\end{aligned}
$$
所以 $ \lim\limits _{k \rightarrow \infty} B \boldsymbol{x}^{(k)}=\boldsymbol{B} \boldsymbol{x}^{*} $.

\end{tcolorbox}

\begin{tcolorbox}[enhanced,colback=green!10!white,colframe=green!50!white,breakable,coltitle=green!25!black,title=2024]
1. 解方程 $ 12-3 x+2 \cos x=0 $ 的迭代格式为 $ x_{n+1}=4+\frac{2}{3} \cos x_{n} $.

(1) 证明: 对任意 $ x_{0} \in \mathbf{R} $, 均有 $ \lim\limits _{n \rightarrow \infty} x_{n}=x^{*} $ ( $ x^{*} $ 为方程的根);

(2) 此迭代法的收敛阶是多少?
 \tcblower
 (1) 迭代函数 $ \varphi(x)=4+\frac{2}{3} \cos x $, 对任意 $ x \in \mathbf{R} $

$$
4-\frac{2}{3} \leqslant 4+\frac{2}{3} \cos x \leqslant 4+\frac{2}{3},\quad x \in(+\infty,-\infty)
$$
$$
\varphi(x) \in\left[4-\frac{2}{3}, 4+\frac{2}{3}\right] \subset(-\infty,+\infty) \\
$$
又因为:
$$
\varphi^{\prime}(x)=-\frac{2}{3} \sin x, \quad L=\max _{-\infty<x<\infty}\left|\varphi^{\prime}(x)\right|=\frac{2}{3}<1
$$

故迭代公式在 $ (-\infty, \infty) $ 满足收敛性定理, 即 $ \left\{x_{k}\right\} $ 收敛于方程的根 $ x^{*} $.

(2) 由
$$
\lim _{k \rightarrow \infty} \frac{x^{*}-x_{k+1}}{x^{*}-x_{k}}=\lim _{k \rightarrow \infty} \frac{\varphi\left(x^{*}\right)-\varphi\left(x_{k}\right)}{x^{*}-x_{k}}=\varphi^{\prime}\left(x^{*}\right)=-\frac{2}{3} \sin x^{*} \neq 0
$$
可知迭代线性收敛.
\end{tcolorbox}

\begin{tcolorbox}[enhanced,colback=blue!8!white,colframe=blue!25!white,breakable,title=2024]

设 $ A $ 为 $ n $ 阶非奇异矩阵且有分解式 $ A=L U $, 其中 $ L $ 是单位下三角阵, $ U $ 为上三角阵, 求证 $ A $ 的所有顺序主子式均不为零.
\tcblower
证明: 由题意可知, 若将 $ A=L U $ 分解式中的 $ L $ 与 $ U $ 分块
$$
L=\left[\begin{array}{cc}
L_{k \times k} & 0_{k \times(n-k)} \\
L_{(n-k) \times k} & L_{(n-k) \times(n-k)}
\end{array}\right], \quad U=\left[\begin{array}{cc}
U_{k \times k} & U_{k \times(n-k)} \\
0_{(n-k) \times k} & U_{(n-k) \times(n-k)}
\end{array}\right]
$$

其中, $ L_{k \times k} $ 为 $ k $ 阶单位下三角阵, $ U_{k \times k} $ 为 $ k $ 阶上三角阵, 则 $ A $ 的 $ k $ 阶顺序主子式为 $ A_{k}=L_{k \times k} U_{k \times k} $, 又由 $ A $ 为 $ n $ 阶非奇异矩阵, 因此 $ |A|=a_{11}^{(1)} a_{22}^{(2)} \cdots a_{n n}^{(n)} \neq 0 $,则
$$
\left|A_{k}\right|=\left|L_{k \times k}\right| \cdot\left|U_{k \times k}\right|=a_{11}^{(1)} a_{22}^{(2)} \cdots a_{k k}^{(k)} \neq 0
$$
证得 $ A $ 的所有顺序主子式均不为零.

\end{tcolorbox}


\begin{tcolorbox}[breakable,enhanced,arc=0mm,outer arc=0mm,
		boxrule=0pt,toprule=1pt,leftrule=0pt,bottomrule=1pt, rightrule=0pt,left=0.2cm,right=0.2cm,
		titlerule=0.5em,toptitle=0.1cm,bottomtitle=-0.1cm,top=0.2cm,
		colframe=white!10!biru,colback=white!90!biru,coltitle=white,
            coltext=black,title =2024-03-05, title style={white!10!biru}, before skip=8pt, after skip=8pt,before upper=\hspace{2em},
		fonttitle=\bfseries,fontupper=\normalsize]
  
4. 为了使 $ \sqrt{11} $ 的近似值的相对误差不超过 $ 0.1 \% $, 至少应取几位有效数字
 \tcblower
 设近似数 $ x^{*} $ 表示为
$$
x^{*}= \pm 10^{m} \times\left(a_{1}+a_{2} \times 10^{-1}+\cdots+a_{l} \times 10^{-(l-1)}\right),
$$
其中 $ a_{i}(i=1,2, \cdots, l) $ 是 0 到 9 中的一个数字, $ a_{1} \neq 0, m $ 为整数. 若 $ x^{*} $ 具有 $ n $ 位有效数字,则其相对误差限
$$
\varepsilon_{r}^{*} \leqslant \frac{1}{2 a_{1}} \times 10^{-(n-1)} 
$$

设取 $ n $ 位有效数字, 由上可知 $ \varepsilon_{r}^{*} \leqslant \frac{1}{2 a_{1}} \times 10^{-n+1} $. 由于 $ \sqrt{11}=3.3\cdots $,知 $ a_{1}=3 $, 

$$
\varepsilon_{r}\left(x^{*}\right) \leqslant \frac{1}{2 a_{1}} \times 10^{1-n}=\frac{1}{6} \times 10^{(1-n)}
$$

根据题意我们有 $ \frac{1}{6} \times 10^{1-n} \leqslant 0.1 \% $, 解得 $ n \geqslant 3.22 $, 故取$n=4$,即只要对 $ \sqrt{11} $ 的近似值取 4 位有效数字, 其相对误差限就小于 $ 0.1 \% $. 
\end{tcolorbox}


\begin{tcolorbox}[breakable,
		colframe=white!10!jingga, 
        coltitle=white!90!jingga, 
        colback=white!95!jingga, 
        coltext=black, 
        colbacktitle=white!10!jingga, 
        enhanced, 
        fonttitle=\bfseries,
        fontupper=\normalsize, 
        attach boxed title to top left={yshift=-2mm}, 
        before skip=8pt, 
        after skip=8pt,
		title=填空题]
 

1. 证明: 求解常微分方程初值问题的改进 Euler 方法具有$\underline{\hspace{1cm}}$阶精度.
    
2. 已知 $ f(2)=3, f(3)=5, f(5)=4 $, 则函数 $ f(x) $ 在此三点的插值多项式为$\underline{\hspace{1cm}}$.

2. 设 $ A=\left(\begin{array}{ll}1 & 1 \\ 0 & 1\end{array}\right) $, 则 $ \operatorname{cond}_{1}(A)= \underline{\hspace{1cm}}$

4. 迭代法 $ X^{(k+1)}=B X^{(k)}+f $ 求解线性方程组对任意 $ X^{(0)} $ 和 $ f $ 均收敛的充要条件为$\underline{\hspace{1cm}}$

5. 定积分的 Simpson 数值求积公式具有$\underline{\hspace{1cm}}$次代数精度.
\end{tcolorbox}



\begin{tcolorbox}[breakable,
		colframe=white!10!jingga, coltitle=white!90!jingga, colback=white!95!jingga, coltext=black, colbacktitle=white!10!jingga, enhanced, fonttitle=\bfseries,fontupper=\normalsize, attach boxed title to top left={yshift=-2mm}, before skip=8pt, after skip=8pt,
		title=解答题]


  证明: 当 $ x_{0}=1.5 $ 时, 迭代法 $ x_{k+1}=\sqrt{\frac{10}{4+x_{k}}} $ 收敛于方程 $ f(x)=x^{3}+4 x^{2}-10=0 $ 在区间 $ [1,2] $ 内唯一实根 $ x^{*} $.

   \tcblower

首先,我们建立迭代公式:
\end{tcolorbox}


\end{document}